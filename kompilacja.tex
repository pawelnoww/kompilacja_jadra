\documentclass[12pt]{report}
\usepackage{polski}
\usepackage{float}

% interlinia
\linespread{1.25}

% marginesy
\usepackage{geometry}
\newgeometry{tmargin=3cm, bmargin=3cm, lmargin=2.5cm, rmargin=2.5cm}

% rysunki
\usepackage{graphicx}
\graphicspath{{figures/}}


% Title Page
\title{Kompilacja jądra}
\author{Paweł Nowak}


\begin{document}
\maketitle


\chapter{Pobieranie ostatniej stabilnej wersji jądra}
Została sprawdzona wersja jądra z użyciem komendy \verb|uname --release|. W systemie znajdowało się jądro w wersji \verb|5.15-27-smp|. Korzystając z witryny \verb*|www.kernel.org| sprawdzono najnowszą stabilną wersję jądra (\verb|5.18.3|).
\begin{figure}[H]
	\centering
	\includegraphics[width=12cm]{1_1}
	\caption{Informacje o jądrze systemu}
\end{figure}

\begin{figure}[H]
	\centering
	\includegraphics[width=12cm]{1_1_2}
	\caption{Fragment witryny www.kernel.org}
\end{figure}

Pobrano najnowszą wersję jądra, uprzednio wchodząc do katalogu \verb|/usr/src|
\begin{figure}[H]
	\centering
	\includegraphics[width=15cm]{1_2}
	\caption{Pobieranie jądra systemu w wersji 5.18.3}
\end{figure}

Rozpakowano pobrane archiwum z użyciem komendy \verb*|tar -xpf linux-5.18.3.tar.xz|
\begin{figure}[H]
	\centering
	\includegraphics[width=15cm]{1_3}
	\caption{Rozpakowywanie archiwum z pobranym jądrem systemu}
\end{figure}


\chapter{Przebieg procesu kompilacji dla starej metody}

Po wejściu w katalog \verb*|linux-5.18.3| wykonano kopię konfiguracji starego jądra. W tym celu wykorzystano komendę \verb*|zcat /proc/config.gz > .config|. Rezultat komendy \verb*|ls -la| potwierdza poprawne skopiowanie pliku konfiguracyjnego (zaznaczono plik na zrzucie ekranu).
\begin{figure}[H]
	\centering
	\includegraphics[width=15cm]{2_1}
	\caption{Kopiowanie pliku konfiguracyjnego}
\end{figure}

Następnie z użyciem komendy \verb*|make localmodconfig| przygotowano plik konfiguracyjny. W każdym pojawiającym się komunikacie ustawiono wartość domyślną danego parametru klikając klawisz \verb*|ENTER|.
\begin{figure}[H]
	\centering
	\includegraphics[width=15cm]{2_2}
	\caption{Rezultat komendy przygotowującej plik konfiguracyjny}
\end{figure}

Przystąpiono do procesu kompilacji jądra. Wykorzystano komendę \verb*|make -j4 bzImage|.
\begin{figure}[H]
	\centering
	\includegraphics[width=15cm]{2_3}
	\caption{Wywołanie komendy rozpoczynającej proces kompilacji jądra}
\end{figure}

Po około 15 minutach proces kompilacji jądra zakończył się pomyślnie. Na standardowym wyjściu widnieje ścieżka do obrazu jądra: \verb*|arch/x86/boot/bzImage|.
\begin{figure}[H]
	\centering
	\includegraphics[width=15cm]{2_4}
	\caption{Końcowy rezultat komendy wywołującej proces kompilacji jądra}
\end{figure}

Kolejno zbudowano moduły jądra. Wywołano komendę \verb*|make -j4 modules|.

\begin{figure}[H]
	\centering
	\includegraphics[width=15cm]{2_5}
	\caption{Wywołanie komendy budującej modułów jądra}
\end{figure}

\begin{figure}[H]
	\centering
	\includegraphics[width=15cm]{2_6}
	\caption{Końcowy rezultat komendy budującej moduły jądra}
\end{figure}

Proces zakończył się pomyślnie po około 2 minutach. Kolejno przystąpiono do procesu instalacji modułów, co wykonano z pomocą komendy \verb*|make -j4 modules install|.

\begin{figure}[H]
	\centering
	\includegraphics[width=15cm]{2_7}
	\caption{Instalacja modułów}
\end{figure}

Kolejnym etapem było kopiowanie plików nowego jądra do katalogu boot z użyciem poniższych komend:
\begin{itemize}
	\item obraz jądra: \verb*|cp arch/x86/boot/bzImage /boot/vmlinuz-method1-5.18.3-smp|
	\item plik konfiguracyjny: \verb*|cp .config /boot/config-method1-5.18.3-smp|
	\item tablica symboli: \verb*|cp System.map /boot/System.map-method1-5.18.3-smp|
\end{itemize}
gdzie \verb*|method1| jest oznaczeniem aktualnie wykonywanej metody kompilacji jądra, a \verb*|5.18.3-smp| wersją jądra.

\begin{figure}[H]
	\centering
	\includegraphics[width=15cm]{2_8}
	\caption{Kopiowanie plików jądra do katalogu boot}
\end{figure}

Kolejnym etapem jest utworzenie linku symbolicznego: do pliku \verb*|System.map| z katalogu \verb*|/boot| należy dołączyć uprzednio skopiowany plik \verb*|System.map-method1-5.18.3-smp|.

\begin{figure}[H]
	\centering
	\includegraphics[width=15cm]{2_9}
	\caption{Tworzenie symbolicznego linku do pliku System.map}
\end{figure}

W celu skonfigurowania RAMDISK, z użyciem skryptu \verb*|mkinitrd_command_generator.sh| wygenerowano stosowną komendę, którą później wykonano: \\
 \verb*|mkinitrd -c -k 5.18.3-smp -f ext4 -r /dev/sda1 -m ext4 -u -o /boot/initrd.gz|.
 W komendzie zmieniono nazwę pliku wyjściowego na taką, która zachowa spójność z poprzednio skopiowanymi plikami jądra.
 \begin{figure}[H]
 	\centering
 	\includegraphics[width=15cm]{2_10}
 	\caption{Tworzenie RAMDISK}
 \end{figure}
 
 Następnie wykonano konfiguracją bootloadera \verb*|LILO|. W tym celu edytowano plik konfiguracyjny \verb*|/etc/lilo.conf| zamieszczając w nim nowy wpis.
  \begin{figure}[H]
  	\centering
  	\includegraphics[width=15cm]{2_11}
  	\caption{Edycja lilo.conf}
  \end{figure}
  Zmiany wprowadzone do pliku konfiguracyjnego zostały zatwierdzone poprzez wywołanie komendy \verb*|lilo|:
    \begin{figure}[H]
    	\centering
    	\includegraphics[width=15cm]{2_12}
    	\caption{Wywołanie komendy lilo}
    \end{figure}
    
    Po zrestartowaniu systemu w boot menu pojawiła się opcja \verb*|method1|. System pomyślnie uruchomił się oraz pomyślnie zalogowano się na użytkownika \verb*|root|.
    
    \begin{figure}[H]
       	\centering
       	\includegraphics[width=15cm]{2_13}
       	\caption{Boot menu po restarcie systemu}
    \end{figure}
        
    \begin{figure}[H]
          	\centering
          	\includegraphics[width=15cm]{2_14}
          	\caption{Pomyślne logowanie oraz start systemu}
    \end{figure}
    
    Po sprawdzeniu wersji jądra otrzymano rezultat \verb*|5.18.3-smp| zgodny z oczekiwaniami.
    \begin{figure}[H]
         	\centering
         	\includegraphics[width=15cm]{2_15}
         	\caption{Aktualna wersja jądra}
    \end{figure}


\chapter{Przebieg procesu kompilacji dla nowej metody}

\begin{figure}[H]
	\centering
	\includegraphics[width=15cm]{1_1}
	\caption{:D}
\end{figure}

\begin{figure}[H]
	\centering
	\includegraphics[width=15cm]{1_1}
	\caption{:D}
\end{figure}

\begin{figure}[H]
	\centering
	\includegraphics[width=15cm]{1_1}
	\caption{:D}
\end{figure}

\begin{figure}[H]
	\centering
	\includegraphics[width=15cm]{1_1}
	\caption{:D}
\end{figure}

\begin{figure}[H]
	\centering
	\includegraphics[width=15cm]{1_1}
	\caption{:D}
\end{figure}

\begin{figure}[H]
	\centering
	\includegraphics[width=15cm]{1_1}
	\caption{:D}
\end{figure}

\begin{figure}[H]
	\centering
	\includegraphics[width=15cm]{1_1}
	\caption{:D}
\end{figure}

\begin{figure}[H]
	\centering
	\includegraphics[width=15cm]{1_1}
	\caption{:D}
\end{figure}

\end{document}          
